\documentclass{article}
\usepackage[utf8]{inputenc}
\usepackage[T1]{fontenc}
\usepackage[MeX]{polski}
\usepackage{graphicx}
\usepackage[outdir=./]{epstopdf}
\usepackage{listings}

\title{Praca dyplomowa magisterska}
\author{Tomasz Wesołowski}
\date{2019}
 
\begin{document}
 
\maketitle
 
\tableofcontents
 
\section{Problem szeregowania zadań}

Szeregowanie zadań kompatybilnych na maszynach jednorodnych.

\section{Kolorowanie grafów dwudzielnych}

\subsection*{Dane wejściowe}

Danymi wejścowymi jest więc graf dwudzielny ważony oraz paleta czterech kolorów z przypisanymi do nich wagami. Wagi kolory układamy niemalejąco.

\subsection*{Weryfikacja poprawności kolorowania}

\lstinputlisting[language=R]{src/is_colored_properly.R}

\subsection*{Koszt kolowania grafu ważonego}

\lstinputlisting[language=R]{src/sum_coloring.R}

\subsection{Algorytm 27/26-przybliżony}

\lstinputlisting[language=R]{src/27_26-approximate-coloring.R}

\subsubsection*{3-pseudo kolorowanie}

\lstinputlisting[language=R]{src/3-pseudocoloring.R}

\subsubsection*{Optymalne 2-kolorowanie}

\lstinputlisting[language=R]{src/2-coloring.R}

\subsection{Algorytm siłowy}

Algorytm siłowy polega na wygenerowaniu wszystkich możliwych kombinacji pokolorowania grafu i sprawdzeniu czy jest to pokolorowanie poprawne. Następnie z wszystkich prawidłowych kolorowań wybieramy to, które ma najmniejszą sumę kolorowania. Ilość wszystkich kombinacji wynosi $n^k$, gdzie $n$ to ilość wierzchołków grafu, a $k$ to ilość różnych kolorów.

\lstinputlisting[language=R]{src/brute_force.R}

\section{Porównanie algorytmów}

Opis metody generowania grafów dwudzielnych ważonych.

\section{Analiza wyników}

\end{document}
